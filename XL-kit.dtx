% \iffalse meta-comment
%
% Copyright (C) 2022-2023 by Jinwen XU
% ------------------------------------
%
% This file may be distributed and/or modified under the conditions of the LaTeX
% Project Public License, either version 1.3c of this license or (at your option)
% any later version. The latest version of this license is in:
%
%    http://www.latex-project.org/lppl.txt
%
% \fi
%
%<*driver>
\ProvidesFile{XL-kit.dtx}
%</driver>
\NeedsTeXFormat{LaTeX2e}[2022-06-01]
%
%<*XL-kit>
\ProvidesExplPackage
  {XL-kit}
  {2023/10/15} {}
  {Collective interface of XL's toolkit}
%</XL-kit>
%
%<*XL-logo>
\ProvidesExplPackage
  {XL-logo}
  {2023/10/15} {}
  {The XL logo}
%</XL-logo>
%
%<*XL-math>
\ProvidesExplPackage
  {XL-math}
  {2023/10/15} {}
  {XL's math configuration}
%</XL-math>
%
%<*XL-text>
\ProvidesExplPackage
  {XL-text}
  {2023/10/15} {}
  {XL's text configuration}
%</XL-text>
%
%<*XL-title>
\ProvidesExplPackage
  {XL-title}
  {2023/10/15} {}
  {XL's title configuration}
%</XL-title>
%
%<*XL-management>
\ProvidesExplPackage
  {XL-management}
  {2023/10/15} {}
  {XL's automatic management for multi-file projects}
%</XL-management>

%<*XL-kit>
\keys_define:nn { XL-kit }
  {
    , title   .bool_set:N = \l__xl_kit__title_enabled_bool
    , title   .initial:n  = { false }

    , unknown .code:n = {
                          \PassOptionsToPackage { \CurrentOption } { ProjLib }
                          \PassOptionsToPackage { \CurrentOption } { XL-logo }
                          \PassOptionsToPackage { \CurrentOption } { XL-math }
                          \PassOptionsToPackage { \CurrentOption } { XL-text }
                          \PassOptionsToPackage { \CurrentOption } { XL-title }
                          \PassOptionsToPackage { \CurrentOption } { XL-management }
                        }
  }
\ProcessKeyOptions [ XL-kit ]

\RequirePackage { ProjLib }
\RequirePackage { XL-logo }
\RequirePackage { XL-math }
\RequirePackage { XL-text }
\RequirePackage { XL-management }

\bool_if:NT \l__xl_kit__title_enabled_bool
  {
    \RequirePackage { XL-title }
  }

\hook_gput_code:nnn { begindocument/before } { XL-kit }
  {
    \file_if_exist_input:n { XL-common }
  }
%</XL-kit>
%
%<*XL-logo>
\keys_define:nn { XL-logo }
  {
    , unknown .code:n = {}
  }
\ProcessKeyOptions [ XL-logo ]

\RequirePackage { projlib-paper }
\RequirePackage { projlib-math }

\RequirePackage { tikz }
\RequirePackage { anyfontsize }

\hook_gput_code:nnn { begindocument/before } { XL-logo }
  {
    \cs_if_exist:cTF { fontspec }
      {
        \newfontface \xl_logo_symbol_font: { LinBiolinum_R.otf }
      }
      {
        \RequirePackage [T1] { fontenc }
        \cs_gset:Nn \xl_logo_symbol_font:
          { \fontfamily { LinuxBiolinumT-TLF } \selectfont }
      }
  }

\NewDocumentCommand \XLlogo {}
  {
    \begin{tikzpicture}
        \clip (0,0) rectangle (1,1);
        \fill[paper,rounded~corners=.05cm] (0,0) rectangle (1,1);
        \node[main-text!33!paper] at (.5,.5) { \fontsize{7}{0}\selectfont\xl_logo_symbol_font: \& };
        \node[main-text!83!paper] at (.25,.75) { \fontsize{14}{0}\selectfont $\mathfrak{X}$ };
        \node[main-text!83!paper] at (.72,.28) { \fontsize{15}{0}\selectfont $\mathfrak{L}$ };
        \fill[blue!50!cyan!75!main-text] (.075,.075) -- (.45,.075) -- (.075,.45);
        \fill[blue!50!cyan!75!main-text] (.925,.925) -- (.55,.925) -- (.925,.55);
    \end{tikzpicture}
  }
%</XL-logo>
%
%<*XL-math>
\keys_define:nn { XL-math }
  {
    , quaternion           .bool_set:N = \l__xl_math_quaternion_bool
    , quaternion           .initial:n  = { false }

    , operator in mathsf   .bool_set:N = \l__xl_math_operator_in_mathsf_bool
    , operator in mathsf   .initial:n  = { false }
    , operator-in-mathsf   .bool_set:N = \l__xl_math_operator_in_mathsf_bool
    , operator~in~mathsf   .bool_set:N = \l__xl_math_operator_in_mathsf_bool
    , operators in mathsf  .bool_set:N = \l__xl_math_operator_in_mathsf_bool
    , operators-in-mathsf  .bool_set:N = \l__xl_math_operator_in_mathsf_bool
    , operators~in~mathsf  .bool_set:N = \l__xl_math_operator_in_mathsf_bool
    , operator sans serif  .bool_set:N = \l__xl_math_operator_in_mathsf_bool
    , operator-sans-serif  .bool_set:N = \l__xl_math_operator_in_mathsf_bool
    , operator~sans~serif  .bool_set:N = \l__xl_math_operator_in_mathsf_bool
    , operators sans serif .bool_set:N = \l__xl_math_operator_in_mathsf_bool
    , operators-sans-serif .bool_set:N = \l__xl_math_operator_in_mathsf_bool
    , operators~sans~serif .bool_set:N = \l__xl_math_operator_in_mathsf_bool

    , fancy functor        .bool_set:N = \l__xl_math_fancy_functor_bool
    , fancy functor        .initial:n  = { true }
    , fancy-functor        .bool_set:N = \l__xl_math_fancy_functor_bool
    , fancy~functor        .bool_set:N = \l__xl_math_fancy_functor_bool
    , fancy functors       .bool_set:N = \l__xl_math_fancy_functor_bool
    , fancy-functors       .bool_set:N = \l__xl_math_fancy_functor_bool
    , fancy~functors       .bool_set:N = \l__xl_math_fancy_functor_bool

    , unknown              .code:n     = {}
  }
\ProcessKeyOptions [ XL-math ]

\RequirePackage { projlib-math }

\RequirePackage { tikz-cd }
\ExplSyntaxOff
\usetikzlibrary{calc}
\ExplSyntaxOn
\RequirePackage { derivative }
\RequirePackage { xfrac }

% Add post- and pre- text to \eqref
\hook_gput_code:nnn { begindocument } { XL-math }
  {
    \cs_if_exist:NT \eqref
      {
        \RenewDocumentCommand \eqref { O{} O{} m }
          {
            \textup { \tagform@ { #2 \ref{#3} #1 } }
          }
      }
  }

% https://tex.stackexchange.com/a/187462
\NewDocumentCommand \nospacepunct { m } { \makebox[0pt][l]{\,#1} }


\NewDocumentCommand \xl_math_decompose_apply:Nnnn { m m m m }
  {
    #1{#2} \IfNoValueF{#3}{\sb{#3}} \IfNoValueF{#4}{\sp{#4}}
  }


% \cs_new:Nn \xl_math_operator_style:n { \mathrm { #1 } }
\cs_new:Nn \xl_math_operator_style:n { \mathsf { #1 } }

%% Currently this only works for unicode-math
\bool_if:NT \l__xl_math_operator_in_mathsf_bool
  {
    % \cs_if_exist_use:N \setoperatorfont \xl_math_operator_style:n
    \cs_if_exist_use:N \setoperatorfont \mathsf
  }

%%==================================================
%%  Common
%%==================================================
\DeclarePairedDelimiter \abs   { \lvert } { \rvert }
\DeclarePairedDelimiter \norm  { \lVert } { \rVert }
\DeclarePairedDelimiter \tnorm { \lVvert } { \rVvert }

% \newcommand*\conj[1]{\overline{#1}}
\NewDocumentCommand \conj { m } { \overline { #1 } }

% https://tex.stackexchange.com/a/118217
\DeclarePairedDelimiter \ceil  { \lceil  } { \rceil  }
\DeclarePairedDelimiter \floor { \lfloor } { \rfloor }

% \DeclarePairedDelimiterX \set [2] \lbrace \rbrace
%   {
%     \mskip1.5mu
%     #1
%     \tl_if_blank:nF { #2 } { \mskip2.5mu \mathbf{\delimsize\vert} \mskip2.5mu #2 }
%     \mskip1.5mu
%   }
% \NewDocumentCommand \setsymbol { O{} }
%   {
%     \vbox_set:Nn \l_tmpa_box {$#1\vert$}
%     \mathrel{
%       \tikz[baseline=\dim_eval:n {\box_dp:N \l_tmpa_box}]
%         \draw[line~width=.12em] (0,0) -- (0,\dim_eval:n {\box_ht_plus_dp:N \l_tmpa_box});
%     }
%     \allowbreak
%     \mathopen{}
%   }
\NewDocumentCommand \setseparator { O{} }
  {
    \mspace { 4mu }
    #1\vert
    \mspace { 4mu }
    \allowbreak
    \mathopen { }
  }
\providecommand \given {}
\DeclarePairedDelimiterX \set [1] \lbrace \rbrace
  {
    \tl_set:Nn \given { \setseparator[\delimsize] }
    \mspace { 1.5mu }
    #1
    \mspace { 1.5mu }
  }

% French style
% \NewDocumentCommand \restrict { m m }
%   {
%     #1 {} \sb { |\mskip1mu #2 }
%   }
%
% https://tex.stackexchange.com/a/22255
% \newcommand\restr[2]{{% we make the whole thing an ordinary symbol
%   \left.\kern-\nulldelimiterspace % automatically resize the bar with \right
%   #1 % the function
%   \vphantom{\big|} % pretend it's a little taller at normal size
%   \right|_{#2} % this is the delimiter
%   }}
%
% \NewDocumentCommand \restrict { m m }
%   {
%     \group_begin:
%       \smash
%         {
%           \left.
%             \kern-\nulldelimiterspace
%             #1
%             \mathchoice
%               { \vphantom { \big| } }  % \displaystyle
%               { \vphantom { \big| } }  % \textstyle
%               { }                      % \scriptstyle
%               { }                      % \scriptscriptstyle
%           \right|
%         }
%       \sb { #2 }
%     \group_end:
%   }
%
\NewDocumentCommand \restrict { s m m }
  {
    \group_begin:
      \smash
        {
          \left.
            \kern-\nulldelimiterspace
            #2
            \mathchoice
              { \vphantom { \big| } }  % \displaystyle
              { \vphantom { \big| } }  % \textstyle
              { }                      % \scriptstyle
              { }                      % \scriptscriptstyle
          \right|
        }
      \IfBooleanT { #1 } { \mathclose { \vphantom { #2 } } }
      \sb { #3 }
    \group_end:
  }

\NewDocumentCommand \singleton {}
  {
    \mspace { 1.5mu }
    \mathord { * }
    \mspace { 1.5mu }
  }

\NewDocumentCommand \genericvariable {}
  {
    \mspace { 3mu }
    \vcenter{\hbox{\rule{.225em}{.225em}}}
    \mspace { 3mu }
  }
\NewDocumentCommand \genericobject {}
  {
    \mspace { 1.5mu }
    \mathord{\square}
    \mspace { 1.5mu }
  }

% https://tex.stackexchange.com/a/217624
\NewDocumentCommand \transpose { m }
  {
    #1 \sp{\@transpose}
  }
\newcommand*{\@transpose}{%
  {\mathpalette\@@transpose{}}%
}
\newcommand*{\@@transpose}[2]{%
  % #1: math style
  % #2: unused
  \raisebox{\depth}{$\m@th#1\intercal$}%
}
% https://tex.stackexchange.com/a/136464
\NewDocumentCommand \franspose { O{-3mu} m }
  {
    \mskip1mu\prescript{\smash{\mathrm t\mkern#1}}{}{\mathstrut#2}
  }

%%==================================================
%%  Constant e and i
%%==================================================
\NewDocumentCommand \e { } { \mathrm{e} }
\RedefineInMathMode \i { \mathrm{i} }
\RedefineInMathMode \j { \mathrm{j} }

%%==================================================
%%  Arrows for morphisms
%%==================================================
\NewDocumentCommand \injection  { } { \hookrightarrow    }
\NewDocumentCommand \surjection { } { \twoheadrightarrow }
\NewDocumentCommand \bijection  { } { \xrightarrow{\sim} }

%%==================================================
%%  Complementary to usual operators
%%==================================================
\DefineMathOperator { sgn }
\DefineMathOperator { lcm }         \DefineMathOperator { pgcd, ppcm }
\DefineMathOperator { ord, card }
\DefineMathOperator { cl }
\DefineMathOperator { characteristic = char }
\DefineMathOperator { grad }
\DefineMathOperator*{ colim }
\DefineMathOperator { diag }
\DefineMathOperator { tr }
\DefineMathOperator { rank }        \DefineMathOperator { rang }
\DefineMathOperator { vecspan = span }
\DefineMathOperator { codim }
\DefineMathOperator { car, disc, res }
\DefineMathOperator { supp }
\DefineMathOperator { proj }

%%==================================================
%%  Operators - Algebra and number theory
%%==================================================
\DefineMathOperator { Ker, Image = Im, Coker, Coim } [ style = \xl_math_operator_style:n ]
\DefineMathOperator { image = im, coker, coim }

\DefineMathOperator { Hom, Aut, End, Isom, Mat, Gal } [ style = \xl_math_operator_style:n ]
\DefineMathOperator { Symm = Sym } [ style = \xl_math_operator_style:n ] % !!
\NewDocumentCommand \HomUL { } { \mathop { \underline { \Hom } } }
\NewDocumentCommand \EndUL { } { \mathop { \underline { \End } } }
\NewDocumentCommand \IsomUL { } { \mathop { \underline { \Isom } } }

\DefineMathOperator { Diag } [ style = \xl_math_operator_style:n ]
\DefineMathOperator { Span, Vect } [ style = \xl_math_operator_style:n ]
\DefineMathOperator { Supp } [ style = \xl_math_operator_style:n ]
\DefineMathOperator { Int } [ style = \xl_math_operator_style:n ]
\DefineMathOperator { Fit } [ style = \xl_math_operator_style:n ]

\DefineMathOperator { Bil, Mult } [ style = \xl_math_operator_style:n ]
\DefineMathOperator { Ob, Obj, Mor, Morph } [ style = \xl_math_operator_style:n ]
\DefineMathOperator { source = s, target = t } [ style = \xl_math_operator_style:n ]
\DefineMathOperator { Tor, tor, Ext, ext } [ style = \xl_math_operator_style:n ]

\DefineMathOperator { Sp, Spec, Spm = SpMax, SpMax, SpecMax, Sph = SpHom, SpHom, SpmHom } [ style = \xl_math_operator_style:n ]
\DefineMathOperator { SpEt = Sp } [ style = \mathfrak ]
\DefineMathOperator { Ann } [ style = \xl_math_operator_style:n ]
\let\Proj\relax % in case babel russian is loaded
\DefineMathOperator { Proj } [ style = \xl_math_operator_style:n ]
\DefineMathOperator { Frac, Princ, Cl } [ style = \xl_math_operator_style:n ]
\DefineMathOperator { Tr, Norm = N } [ style = \xl_math_operator_style:n ]
\DefineMathOperator { D, Disc, Res } [ style = \xl_math_operator_style:n ]
\DefineMathOperator { Pic } [ style = \xl_math_operator_style:n ]
\DefineMathOperator { Ind } [ style = \xl_math_operator_style:n ]

\DefineMathOperator { Vanish = V, Ideal = I, ZeroSet = Z } [ style = \xl_math_operator_style:n ]

\DefineMathOperator { Stab, Fix, Orb, Per } [ style = \xl_math_operator_style:n ]
\DefineMathOperator { GL, SL, SU, SO } [ style = \xl_math_operator_style:n ]

\DefineMathOperator { gl, su, so } [ style = \mathfrak ]
\RedefineInMathMode \sl { \mathfrak{sl} }

\DefineMathOperator { Der, Jac } [ style = \xl_math_operator_style:n ]

\DefineMathOperator { gr } [ style = \xl_math_operator_style:n ]

% \NewDocumentCommand \dimkrull { } { \dim\sb{\mathrm{Krull}} }
\NewDocumentCommand \dimkrull { } { \dim\sb{\text{\tiny$\mathrm{Krull}$}} }

%%==================================================
%%  Symbols - Common
%%==================================================
\DefineMathSymbol { id, Id, 1 } [ style = \xl_math_operator_style:n ]

\DefineMathSymbol { ev } [ style = \xl_math_operator_style:n ]
\DefineMathSymbol { pr } [ style = \xl_math_operator_style:n ]
\DefineMathSymbol { I } [ style = \xl_math_operator_style:n ]

% \DefineMathSymbol { Rel = R } [ style = \mathscr ]
\NewDocumentCommand \Rel {} { \mathbin { \mathscr { R } } }

\RedefineInMathMode \L { \mathcal{L} }
\DefineMathSymbol { Lin = L } [ style = \mathcal ]
\DefineMathSymbol { Sym = S, Alt = A } [ style = \mathfrak ]
\DefineMathSymbol { FT = F } [ style = \mathcal ]
\DefineMathSymbol { Frob = F } [ style = \mathscr ]
\DefineMathSymbol { Scindage = S } [ style = \mathscr ]

%%==================================================
%%  Symbols - Sets
%%==================================================
\RedefineInMathMode \P { \mathbb{P} }
\RedefineInMathMode \S { \mathbb{S} }

\DefineMathSymbol { N, Z, Q, R, C, K, F, A, TT = T, G } [ style = \mathbb ]

\RedefineInMathMode \O { \mathcal{O} }
\DefineMathSymbol { M } [ style = \mathcal ]

\bool_if:NTF \l__xl_math_quaternion_bool
  {
    \RedefineInMathMode \k { \mathrm{k} }
    \NewDocumentCommand \kk { } { \Bbbk }
    \RedefineInMathMode \H { \mathbb{H} }
    \DefineMathSymbol { HH = H } [ style = \mathsf ] % For (co)homologies
  }
  {
    \RedefineInMathMode \k { \Bbbk }
    \RedefineInMathMode \H { \mathsf{H} } % For (co)homologies
  }

%%==================================================
%%  Symbols - Categories
%%==================================================
\ProvideDocumentCommand \category { m }
  {
    \mathcal { #1 }
  }
\DefineMathSymbol { CC = C, DD = D, EE = E, FF = F, II = I, JJ = J } [ style = \category ]

\NewDocumentCommand \CategoryFont { m }
  {
    \tl_set:Nn \l_tmpa_tl {#1}
    \tl_replace_all:Nnn \l_tmpa_tl { ~ } { }
    \text{\normalfont\textbf{\textsf{\l_tmpa_tl}}}
  }
\DefineMathSymbol [ style = \CategoryFont ]
  {
    Set, Mono, Gp, Grp, Ab, Ring, Rng, Field, Grad,
    cVect = Vect, Mod, Alg, Nul,
    Cone,
    Top, Comp, TopPt, RTop, Ouv,
    Sheaf, Presheaf, PreSheaf,
    Sh, PSh,
    RS, RingedSpace, LRS, LocallyRingedSpace,
    Var, Sch, Aff,
    QCoh, Coh,
    Meas,
  }
\NewDocumentCommand \lMod { } { \nobreak\text{\normalfont-}\nobreak \Mod }
\NewDocumentCommand \rMod { } { \Mod \nobreak\text{\normalfont-}\nobreak }
\NewDocumentCommand \lAlg { } { \nobreak\text{\normalfont-}\nobreak \Alg }
\NewDocumentCommand \rAlg { } { \Alg \nobreak\text{\normalfont-}\nobreak }
\NewDocumentCommand \lSet { } { \nobreak\text{\normalfont-}\nobreak \Set }
\NewDocumentCommand \rSet { } { \Set \nobreak\text{\normalfont-}\nobreak }
\NewDocumentCommand \lEns { } { \nobreak\text{\normalfont-}\nobreak \Ens }
\NewDocumentCommand \rEns { } { \Ens \nobreak\text{\normalfont-}\nobreak }
\NewDocumentCommand \lVect { } { \nobreak\text{\normalfont-}\nobreak \cVect }
\NewDocumentCommand \lVar { } { \nobreak\text{\normalfont-}\nobreak \Var }
\NewDocumentCommand \lSch { } { \nobreak\text{\normalfont-}\nobreak \Sch }
\NewDocumentCommand \lAff { } { \nobreak\text{\normalfont-}\nobreak \Aff }

% French version
\DefineMathSymbol [ style = \CategoryFont ]
  {
    Ens, Anneaux = Ann, Prefaisceaux = Pref, Faisceaux = Faisc, EspAnn, EspLocAnn
  }

\RedefineInMathMode \o { \mathrm{o} }
\DefineMathSymbol { op, opp }

\bool_if:NTF \l__xl_math_fancy_functor_bool
  {
    \cs_if_exist:NTF \mathsfit
      {
        \NewDocumentCommand \functor { m } { \mathsfit { #1 } }
      }
      {
        \NewDocumentCommand \functor { m } { #1 }
      }
  }
  {
    \NewDocumentCommand \functor { m } { #1 }
  }

%%==================================================
%%  Misc
%%==================================================
\NewDocumentCommand \ideal { m } { \xl_math_ideal_aux:w #1 }
\NewDocumentCommand \xl_math_ideal_aux:w { m e{_^} }{ \xl_math_decompose_apply:Nnnn \mathfrak { #1 } { #2 } { #3 } }

\let\sheafstyle\mathscr
\NewDocumentCommand \sheaf { m } { \xl_math_sheaf_aux:w #1 }
\NewDocumentCommand \xl_math_sheaf_aux:w { m e{_^} }{ \xl_math_decompose_apply:Nnnn \sheafstyle { #1 } { #2 } { #3 } }

% \NewDocumentCommand \ideal { m } { \mathfrak { #1 } }
% \NewDocumentCommand \sheaf { m } { \mathscr { #1 } }

\NewDocumentCommand \rationalmap { O{} }
  {
    \mathrel{\tikz [line~width = .6pt, >={Stealth[round,length=3.4pt,width=6.15pt,inset=2.25pt]}, ->, dash~pattern=on 2pt off 2pt,#1] \draw [#1] (0pt,0.5ex) -- (1em,0.5ex);}
  }
\NewDocumentCommand \pointo { O{} }
  {
    \mathrel{\tikz [line~width = .6pt, >={Stealth[round,length=3.4pt,width=6.15pt,inset=2.25pt]}, ->, dash~pattern=on .8pt off 1.2pt,#1] \draw [#1] (0pt,0.5ex) -- (1em,0.5ex);}
  }

% \newcommand{\bigast}{\mathop{\scalebox{1.5}{\raisebox{-0.2ex}{$\ast$}}}}
\NewDocumentCommand \bigast { }
  {
    \mathop
      {
        \mathchoice
          { \scalebox{3}{\raisebox{-0.4ex}{$\ast$}} }    % \displaystyle
          { \scalebox{2}{\raisebox{-0.27ex}{$\ast$}} }   % \textstyle
          { \scalebox{1.5}{\raisebox{-0.2ex}{$\ast$}} }  % \scriptstyle
          { \scalebox{1.2}{\raisebox{-0.15ex}{$\ast$}} } % \scriptscriptstyle
      }
  }

\newcommand{\svdots}{%
  \vbox {\baselineskip 3\p@ \kern 3\p@ \hbox{.}\hbox{.}\hbox{.}}
}
% https://tex.stackexchange.com/a/594688
\newcommand{\ssvdots}{%
  \vbox{\fontsize{\sf@size}{\sf@size pt}\linespread{0.3}\selectfont
    \kern0.2\baselineskip
    \hbox{.}\hbox{.}\hbox{.}%
    \kern0.1\baselineskip
  }%
}

% https://tex.stackexchange.com/a/39222
\NewCommandCopy\@@pmod\pmod
\DeclareRobustCommand{\pmod}{\@ifstar\@pmods\@@pmod}
\def\@pmods#1{\mkern4mu({\operator@font mod}\mkern 6mu#1)}

% https://tex.stackexchange.com/a/433881
\newcommand{\genlegendre}[4]{%
  \genfrac{(}{)}{}{#1}{#3}{#4}%
  \if\relax\detokenize{#2}\relax\else\sb{\!#2}\fi
}
\newcommand{\legendre}[3][]{\genlegendre{}{#1}{\,#2\,}{\,#3\,}}
\newcommand{\dlegendre}[3][]{\genlegendre{0}{#1}{\,#2\,}{\,#3\,}}
\newcommand{\tlegendre}[3][]{\genlegendre{1}{#1}{\,#2\,}{\,#3\,}}
%</XL-math>
%
%<*XL-text>
\keys_define:nn { XL-text }
  {
    , unknown .code:n = {}
  }
\ProcessKeyOptions [ XL-text ]

\RequirePackage { projlib-text }
\RequirePackage { needspace }
\RequirePackage { tabto }

% \NewCommandCopy \Liscription \ItemDescription
\ProvideDocumentCommand \Liscription { d() }
  {
    \textcolor { main-text!50!paper } { $($ #1 $)$  }
    \nobreakspace
    \ignorespaces
  }

\RequirePackage { aliascnt }
\cs_if_exist:cT { c@chapter }
  {
    \newaliascnt { mainstream   } { chapter }
    \NewCommandCopy \mainstream    \chapter
  }
\newaliascnt { stream       } { section }
\newaliascnt { substream    } { subsection }
\newaliascnt { subsubstream } { subsubsection }
\NewCommandCopy \stream        \section
\NewCommandCopy \substream     \subsection
\NewCommandCopy \subsubstream  \subsubsection

\cs_if_exist:cT { sectionsub }
  {
    \newaliascnt { streamsub } { sectionsub }
    \NewCommandCopy \streamsub  \sectionsub
  }
\cs_if_exist:cT { subsectionsub }
  {
    \newaliascnt { substreamsub } { subsectionsub }
    \NewCommandCopy \substreamsub  \subsectionsub
  }


\RequirePackage { enumitem }

\newlist{enumerate-alph}{enumerate}{10}
\setlist[enumerate-alph]{leftmargin = \maxof{\parindent}{1.5em},
    label = \normalfont\emph{\alph*}$\mskip-.5mu\big)$,
    ref   = \normalfont\color{.!45!paper}\emph{\alph*}$\mskip-.5mu\big)$}
\newlist{enumerate-Alph}{enumerate}{10}
\setlist[enumerate-Alph]{leftmargin = \maxof{\parindent}{1.5em},
    labelsep = .3em, % Alph numerals are wider so one has to reduce the \labelsep
    label = \normalfont\parbox[t]{1em}{\centering\Alph*}$\mskip-3mu\big)$,
    ref   = \normalfont\color{.!45!paper}\settowidth{\dimen0}{\Alph*}\hspace*{\dimexpr.15ex+(\dimen0-1em)/2\relax}\parbox[t]{1em}{\centering\Alph*}$\mskip-3mu\big)$}
\newlist{enumerate-arabic}{enumerate}{10}
\setlist[enumerate-arabic]{leftmargin = \maxof{\parindent}{1.5em},
    label = \normalfont\arabic*$\mskip-.5mu\big)$,
    ref   = \normalfont\color{.!45!paper}\arabic*$\mskip-.5mu\big)$}
\newlist{enumerate-roman}{enumerate}{10}
\setlist[enumerate-roman]{leftmargin = \maxof{\parindent}{1.5em},
    labelsep = .3em, % roman numerals are wider so one has to reduce the \labelsep
    label = \normalfont\roman*$\mskip-.5mu\big)$,
    ref   = \normalfont\color{.!45!paper}\roman*$\mskip-.5mu\big)$}
\newlist{enumerate-Roman}{enumerate}{10}
\setlist[enumerate-Roman]{leftmargin = \maxof{\parindent}{3em},
    label = \normalfont\Roman*$\mskip-.5mu\big)$,
    ref   = \normalfont\color{.!45!paper}\Roman*$\mskip-.5mu\big)$}

\newlist{subenumerate}{enumerate}{1}
\setlist[subenumerate]{
    label = \normalfont\arabic{enumi}\emph{\alph*}$\mskip-.5mu\big)$,
    ref   = \normalfont\color{.!45!paper}\arabic{enumi}\emph{\alph*}$\mskip-.5mu\big)$}


\ProvideDocumentCommand \manualitem { m }
  {
    \item [ {#1} ]
    \tl_set:Nn \@currentlabel { #1 }
  }

% ibid.
\ProvideDocumentCommand { \ibid } { }
  {
    \projlib_text_abbreviation_generic:nn { \textit{ibid.} } { }
  }

% et seq.
\ProvideDocumentCommand { \etseq } { }
  {
    \projlib_text_abbreviation_generic:nn { \textit{et \nobreakspace seq.} } { }
  }

% loc. cit.
\ProvideDocumentCommand { \loccit } { }
  {
    \projlib_text_abbreviation_generic:nn { \textit{loc. \nobreak \hskip.4ex \nobreak cit.} } { }
  }
%</XL-text>
%
%<*XL-title>
\keys_define:nn { XL-title }
  {
    , unknown .code:n = {}
  }
\ProcessKeyOptions [ XL-title ]

\RequirePackage { XL-logo }

\providecommand{\LocallyStopLineNumbers}{}
\providecommand{\ResumeLineNumbers}{}

%%==================================================
%%  Redefine title style
%%==================================================

\renewcommand{\@maketitle}
  {
    \LocallyStopLineNumbers

    \dim_set:Nn \l_tmpa_dim { \parskip }

    \begin{tcolorbox}[
      enhanced~jigsaw, sharp~corners,
      sidebyside,
      % sidebyside~adapt=left,
      lefthand~width = .33\linewidth,
      lower~separated = false,
      toprule = .75pt, bottomrule = .75pt,
      leftrule = 0pt, rightrule = 0pt,
      boxsep = 0pt,
      top = 1.5ex, bottom = 1.5ex,
      left = 0pt, right = 0pt,
      colframe = main-text!27!paper,
      colback = paper,
    ]
      \scalebox{
        \fp_eval:n { \dim_to_fp:n { 3em } / \dim_to_fp:n { 1cm } }
      }{\XLlogo}
      \tcblower
      \begin{flushright}
        \let\footnote\thanks
        {\bfseries\sffamily\@title}\\\medskip
        \color{main-text!80!paper}
        {\small\@author}
        \par\vspace{-\parskip}\vspace{2pt}
        {\small\@date}
      \end{flushright}
      \dim_compare:nNnT { \l_tmpa_dim } < { 2pt } { \vspace{-.8\baselineskip} }
      \vspace{-2pt}
    \end{tcolorbox}
    \vspace{\parskip}

    \par
    \ResumeLineNumbers
  }

\hook_gput_code:nnn { begindocument/end } { XL-title }
  {
    \tl_if_empty:NF \@title
      {
        \maketitle
      }
  }

%%================================
%%  Redefine abstract style
%%================================

\renewenvironment{abstract}
  {
    \LocallyStopLineNumbers
    \begin{flushright}
      {\bfseries\sffamily\small\abstractname}\par
      \vspace{-\parskip}
      \vspace{-.25\baselineskip}
      \begin{minipage}[t]{.833\textwidth}
        \vspace{0pt}
        \color{main-text!80!paper}
        \footnotesize
        \parindent=2em
  }
  {
      \end{minipage}
    \end{flushright}
    \bigskip
    \ResumeLineNumbers
  }

%% The following code is for the compatibility with the document class
\RequirePackage { projlib-author }

\RenewCommandCopy \projlib_author_backup_abstract: \abstract
\RenewCommandCopy \projlib_author_backup_endabstract: \endabstract
\cs_undefine:N \abstract
\cs_undefine:N \endabstract

\RequirePackage { scontents }
\cs_undefine:N \__scontents_abstract_env_begin:
\cs_undefine:N \__scontents_abstract_env_end:
\newenvsc { abstract } [ store-env = projlib-author-abstract, print-env = false ]
%</XL-title>
%
%<*XL-management>
\keys_define:nn { XL-management }
  {
    , sub-file .bool_set:N = \g__xl_management_subfile_bool
    , sub-file .initial:n  = { false }
    , sub~file .bool_set:N = \g__xl_management_subfile_bool
    , sub file .bool_set:N = \g__xl_management_subfile_bool

    , adjust    .clist_set:N = \g__xl_management_adjustment_mode_clist
    , adjust-to .clist_set:N = \g__xl_management_adjustment_mode_clist
    , adjust~to .clist_set:N = \g__xl_management_adjustment_mode_clist
    , adjust to .clist_set:N = \g__xl_management_adjustment_mode_clist

    , unknown  .code:n = {}
  }
\ProcessKeyOptions [ XL-management ]

\RequirePackage { docmute }


\ior_new:N \l__xl_management_file_ior

% When subfile mode is not on, detect if it is actually a subfile
\bool_if:NF \g__xl_management_subfile_bool
  {
    \file_if_exist:nT { ../list.txt }
      {
        \ior_open:Nn \l__xl_management_file_ior { ../list.txt }
        \ior_map_inline:Nn \l__xl_management_file_ior
          {
            \exp_args:No \str_if_eq:nnT { \jobname } { #1 }
              {
                \bool_gset_true:N \g__xl_management_subfile_bool
              }
          }
        \ior_close:N \l__xl_management_file_ior
      }
  }

% Load common settings for main and sub-files
\bool_if:NTF \g__xl_management_subfile_bool
  {
    \ExplSyntaxOff
    \InputIfFileExists{../common-settings.tex}{}{}
    \ExplSyntaxOn
  }
  {
    \ExplSyntaxOff
    \InputIfFileExists{common-settings.tex}{}{}
    \ExplSyntaxOn
  }

% Allow referencing across files
\bool_if:NT \g__xl_management_subfile_bool
  {
    \RequirePackage { xr }
    \ior_open:Nn \l__xl_management_file_ior { ../list.txt }
    \ior_map_inline:Nn \l__xl_management_file_ior
      {
        \exp_args:No \str_if_eq:nnF { \jobname } { #1 }
          {
            \ExplSyntaxOff
            \externaldocument{../#1/#1}
            \ExplSyntaxOn
          }
      }
    \ior_close:N \l__xl_management_file_ior
  }

\NewDocumentCommand \Input { m }
  {
    \input { #1/#1.tex }
  }

\NewDocumentCommand \Include { m }
  {
    \include { #1/#1.tex }
  }

% Add index to subfiles
\bool_if:NT \g__xl_management_subfile_bool
  {
    \hook_gput_code:nnn { enddocument } { XL-management } { \printindex }
  }


% Short and detailed ToC
\RequirePackage { projlib-language }
\RequirePackage { titletoc }
\hook_gput_code:nnn { begindocument } { XL-management }
  {
    \startcontents[XL-short]
    \startcontents[XL-detailed]
  }

\def\shortcontentsname{\contentsname}
% \def\@startshorttoc#1{\bgroup
%   \makeatletter
%   \@input{\jobname.#1}%
%   \global\@nobreakfalse \egroup}
% \NewDocumentCommand \shorttableofcontents { O{1} }
%   {
%     \bgroup
%     \setcounter{tocdepth}{#1}
%     \@restonecolfalse
%     \if@twocolumn\@restonecoltrue\onecolumn\fi
%     \@ifundefined{chapter}%
%     {\section*{{\shortcontentsname}
%         \@mkboth{\uppercase{\shortcontentsname}}{\uppercase{\shortcontentsname}}}}%
%     {\chapter*{{\shortcontentsname}
%         \@mkboth{\uppercase{\shortcontentsname}}{\uppercase{\shortcontentsname}}}}%
%     \pdfbookmark[1]{\contentsname}{toc}%
%     \@startshorttoc{toc}\if@restonecol\twocolumn\fi
%     \egroup
%   }
\NewDocumentCommand \shorttableofcontents { O{1} }
  {
    \@ifundefined{c@chapter}{}{\cleardoublepage}
    \phantomsection
    \pdfbookmark[1]{\shortcontentsname}{toc}%
    \@ifundefined{c@chapter}%
      {\section*{{\shortcontentsname}
          \@mkboth{\uppercase{\shortcontentsname}}{\uppercase{\shortcontentsname}}}}%
      {\chapter*{{\shortcontentsname}
          \@mkboth{\uppercase{\shortcontentsname}}{\uppercase{\shortcontentsname}}}}%
    \bgroup
    \parskip=0pt
    \printcontents[XL-short]{}{-1}[#1]{}
    \egroup
    \@ifundefined{c@chapter}{}{\cleardoublepage}
  }

\projlib_langauge_define_multilingual_text:Nn \detailedcontentsname
  {
    , EN = Detailed~table~of~contents
    , FR = Table~des~matières~détaillée
    , DE = Detailliertes~Inhaltsverzeichnis
    , IT = Indice~dettagliato
    , PT = Conteúdo~detalhado
    , BR = Sumário~detalhado
    , ES = Índice~detallado
    , CN = 目录(详细)
    , TC = 目錄(詳細)
    , JP = 目次(詳細)
    , RU = Оглавление~(подробно)
  }
% \NewDocumentCommand \detailedtableofcontents { O{3} }
%   {
%     \setcounter{tocdepth}{#1}
%     \bgroup
%     \let\contentsname\detailedcontentsname
%     \tableofcontents
%     \egroup
%   }
\NewDocumentCommand \detailedtableofcontents { O{3} }
  {
    \bookmarksetup{startatroot}
    \stopcontents[XL-detailed]
    \@ifundefined{c@chapter}{}{\cleardoublepage}
    \phantomsection
    \@ifundefined{c@chapter}%
      {\addcontentsline{toc}{section}{\detailedcontentsname}%
        \section*{{\detailedcontentsname}
          \@mkboth{\uppercase{\detailedcontentsname}}{\uppercase{\detailedcontentsname}}}}%
      {\addcontentsline{toc}{chapter}{\detailedcontentsname}%
        \chapter*{{\detailedcontentsname}
          \@mkboth{\uppercase{\detailedcontentsname}}{\uppercase{\detailedcontentsname}}}}%
    \bgroup
    \parskip=0pt
    \printcontents[XL-detailed]{}{-1}[#1]{}
    \egroup
    \@ifundefined{c@chapter}{}{\cleardoublepage}
    \resumecontents[XL-detailed]
  }


% An unified command for adjustments
\msg_new:nnn { xl-management }
  { unknown-adjustment-type }
  { Unknown~adjustment~type~"\l_keys_key_str". }

\keys_define:nn { xl-management-adjust }
  {
    , hspace           .code:n       = { \hspace { #1 } }
    , hspace*          .code:n       = { \hspace*{ #1 } }
    , vspace           .code:n       = { \vspace { #1 } }
    , vspace*          .code:n       = { \vspace*{ #1 } }
    , smallskip        .code:n       = { \smallskip }
    , medskip          .code:n       = { \medskip   }
    , bigskip          .code:n       = { \bigskip   }

    , above            .code:n       = { \setlength { \abovedisplayskip } { #1 } } % \setlength { \abovedisplayshortskip } { #1 }
    , below            .code:n       = { \setlength { \belowdisplayskip } { #1 } } % \setlength { \belowdisplayshortskip } { #1 }
    , above and below  .code:n       = { \setlength { \abovedisplayskip } { #1 }
                                         \setlength { \belowdisplayskip } { #1 } }
    , above-and-below  .meta:n       = { above and below = #1 }
    , above~and~below  .meta:n       = { above and below = #1 }

    , enlargethispage  .code:n       = { \enlargethispage { #1 } }
    , enlargethispage* .code:n       = { \enlargethispage*{ #1 } }
    , enlarge          .meta:n       = { enlargethispage = #1 }
    , enlarge*         .meta:n       = { enlargethispage*= #1 }

    , clearpage        .code:n       = { \clearpage }
    , clear            .meta:n       = { clearpage }
    , pagebreak        .code:n       = { \pagebreak }
    , break            .meta:n       = { pagebreak }

    , par              .code:n       = { \par }
    , linebreak        .code:n       = { \linebreak }

    , code             .code:n       = { #1 }

    , unknown          .code:n       = { \msg_warning:nnn { xl-management } { unknown-adjustment-type } { #1 } }
  }

\clist_if_exist:NF \g__xl_management_adjustment_mode_clist
  {
    \clist_new:N \g__xl_management_adjustment_mode_clist
  }
\NewDocumentCommand \adjust { O{} +m }
  {
    \tl_if_blank:nTF { #1 }
      {
        \keys_set:nn { xl-management-adjust } { #2 }
      }
      {
        \clist_if_in:NnT \g__xl_management_adjustment_mode_clist { #1 }
          {
            \keys_set:nn { xl-management-adjust } { #2 }
          }
      }
  }
%</XL-management>

\endinput